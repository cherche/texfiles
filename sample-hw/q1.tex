\question{1}{a}
My solution to the first question is pretty great.

The following result immediately follows.

\begin{theorem}[Big Mouth Theorem]
  I have a big mouth.
\end{theorem}
\begin{proof}
  Look at the size of it!
\end{proof}
\begin{corollary}
  I can eat a lot.
\end{corollary}


\question{1}{b}
The solution is left as an exercise to the grader.


\question{1}{d} \label{q1d}
We shall solve \ref{q1d} before \ref{q1c}.
\begin{eg}
  The result holds for \(n = 1\).
\end{eg} \label{eg:1}
\begin{proof}
  This is trivially true.
\end{proof}
By cheating, the above example immediately generalizes.
\begin{prop}
  The result holds for \(n \in \N\).
\end{prop}
\begin{proof}
  Simply extrapolate from \ref{eg:1}.
\end{proof}
The above proof illustrates an important technique: letting things go. Occasionally, when the details aren't too important and you don't care about being correct, you can just be lazy and probably wrong.

From the above proposition, the result follows.


\question{1}{c} \label{q1c}
Now \ref{q1c} follows immediately from \ref{q1d}.
