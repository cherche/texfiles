\question{1}{a}
My solution to the first question is pretty great

\begin{theorem}
  I have a big mouth.
\end{theorem}
\begin{proof}
  It's big.
\end{proof}
\begin{corollary}
  I can eat a lot.
\end{corollary}


\question{1}{b}
The solution is left as an exercise to the grader.


\question{1}{d} \label{q1d}
We shall solve \ref{q1d} before \ref{q1c}.
\begin{prop}
  It works for \(n \in \N\).
\end{prop}
\begin{proof}
  Try it for \(n = 1\) and extrapolate like an engineer.
\end{proof}
The above proof illustrates an important technique: letting things go. Occasionally, when the details aren't too important and you don't care about being correct, you can just be lazy and probably wrong.

From the above result, the result follows.


\question{1}{c} \label{q1c}
Now \ref{q1c} follows immediately from \ref{q1d}.
